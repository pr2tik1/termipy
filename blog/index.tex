% Options for packages loaded elsewhere
\PassOptionsToPackage{unicode}{hyperref}
\PassOptionsToPackage{hyphens}{url}
\PassOptionsToPackage{dvipsnames,svgnames,x11names}{xcolor}
%
\documentclass[
  letterpaper,
  DIV=11,
  numbers=noendperiod]{scrreprt}

\usepackage{amsmath,amssymb}
\usepackage{iftex}
\ifPDFTeX
  \usepackage[T1]{fontenc}
  \usepackage[utf8]{inputenc}
  \usepackage{textcomp} % provide euro and other symbols
\else % if luatex or xetex
  \usepackage{unicode-math}
  \defaultfontfeatures{Scale=MatchLowercase}
  \defaultfontfeatures[\rmfamily]{Ligatures=TeX,Scale=1}
\fi
\usepackage{lmodern}
\ifPDFTeX\else  
    % xetex/luatex font selection
\fi
% Use upquote if available, for straight quotes in verbatim environments
\IfFileExists{upquote.sty}{\usepackage{upquote}}{}
\IfFileExists{microtype.sty}{% use microtype if available
  \usepackage[]{microtype}
  \UseMicrotypeSet[protrusion]{basicmath} % disable protrusion for tt fonts
}{}
\makeatletter
\@ifundefined{KOMAClassName}{% if non-KOMA class
  \IfFileExists{parskip.sty}{%
    \usepackage{parskip}
  }{% else
    \setlength{\parindent}{0pt}
    \setlength{\parskip}{6pt plus 2pt minus 1pt}}
}{% if KOMA class
  \KOMAoptions{parskip=half}}
\makeatother
\usepackage{xcolor}
\usepackage{svg}
\setlength{\emergencystretch}{3em} % prevent overfull lines
\setcounter{secnumdepth}{5}
% Make \paragraph and \subparagraph free-standing
\makeatletter
\ifx\paragraph\undefined\else
  \let\oldparagraph\paragraph
  \renewcommand{\paragraph}{
    \@ifstar
      \xxxParagraphStar
      \xxxParagraphNoStar
  }
  \newcommand{\xxxParagraphStar}[1]{\oldparagraph*{#1}\mbox{}}
  \newcommand{\xxxParagraphNoStar}[1]{\oldparagraph{#1}\mbox{}}
\fi
\ifx\subparagraph\undefined\else
  \let\oldsubparagraph\subparagraph
  \renewcommand{\subparagraph}{
    \@ifstar
      \xxxSubParagraphStar
      \xxxSubParagraphNoStar
  }
  \newcommand{\xxxSubParagraphStar}[1]{\oldsubparagraph*{#1}\mbox{}}
  \newcommand{\xxxSubParagraphNoStar}[1]{\oldsubparagraph{#1}\mbox{}}
\fi
\makeatother

\usepackage{color}
\usepackage{fancyvrb}
\newcommand{\VerbBar}{|}
\newcommand{\VERB}{\Verb[commandchars=\\\{\}]}
\DefineVerbatimEnvironment{Highlighting}{Verbatim}{commandchars=\\\{\}}
% Add ',fontsize=\small' for more characters per line
\usepackage{framed}
\definecolor{shadecolor}{RGB}{241,243,245}
\newenvironment{Shaded}{\begin{snugshade}}{\end{snugshade}}
\newcommand{\AlertTok}[1]{\textcolor[rgb]{0.68,0.00,0.00}{#1}}
\newcommand{\AnnotationTok}[1]{\textcolor[rgb]{0.37,0.37,0.37}{#1}}
\newcommand{\AttributeTok}[1]{\textcolor[rgb]{0.40,0.45,0.13}{#1}}
\newcommand{\BaseNTok}[1]{\textcolor[rgb]{0.68,0.00,0.00}{#1}}
\newcommand{\BuiltInTok}[1]{\textcolor[rgb]{0.00,0.23,0.31}{#1}}
\newcommand{\CharTok}[1]{\textcolor[rgb]{0.13,0.47,0.30}{#1}}
\newcommand{\CommentTok}[1]{\textcolor[rgb]{0.37,0.37,0.37}{#1}}
\newcommand{\CommentVarTok}[1]{\textcolor[rgb]{0.37,0.37,0.37}{\textit{#1}}}
\newcommand{\ConstantTok}[1]{\textcolor[rgb]{0.56,0.35,0.01}{#1}}
\newcommand{\ControlFlowTok}[1]{\textcolor[rgb]{0.00,0.23,0.31}{\textbf{#1}}}
\newcommand{\DataTypeTok}[1]{\textcolor[rgb]{0.68,0.00,0.00}{#1}}
\newcommand{\DecValTok}[1]{\textcolor[rgb]{0.68,0.00,0.00}{#1}}
\newcommand{\DocumentationTok}[1]{\textcolor[rgb]{0.37,0.37,0.37}{\textit{#1}}}
\newcommand{\ErrorTok}[1]{\textcolor[rgb]{0.68,0.00,0.00}{#1}}
\newcommand{\ExtensionTok}[1]{\textcolor[rgb]{0.00,0.23,0.31}{#1}}
\newcommand{\FloatTok}[1]{\textcolor[rgb]{0.68,0.00,0.00}{#1}}
\newcommand{\FunctionTok}[1]{\textcolor[rgb]{0.28,0.35,0.67}{#1}}
\newcommand{\ImportTok}[1]{\textcolor[rgb]{0.00,0.46,0.62}{#1}}
\newcommand{\InformationTok}[1]{\textcolor[rgb]{0.37,0.37,0.37}{#1}}
\newcommand{\KeywordTok}[1]{\textcolor[rgb]{0.00,0.23,0.31}{\textbf{#1}}}
\newcommand{\NormalTok}[1]{\textcolor[rgb]{0.00,0.23,0.31}{#1}}
\newcommand{\OperatorTok}[1]{\textcolor[rgb]{0.37,0.37,0.37}{#1}}
\newcommand{\OtherTok}[1]{\textcolor[rgb]{0.00,0.23,0.31}{#1}}
\newcommand{\PreprocessorTok}[1]{\textcolor[rgb]{0.68,0.00,0.00}{#1}}
\newcommand{\RegionMarkerTok}[1]{\textcolor[rgb]{0.00,0.23,0.31}{#1}}
\newcommand{\SpecialCharTok}[1]{\textcolor[rgb]{0.37,0.37,0.37}{#1}}
\newcommand{\SpecialStringTok}[1]{\textcolor[rgb]{0.13,0.47,0.30}{#1}}
\newcommand{\StringTok}[1]{\textcolor[rgb]{0.13,0.47,0.30}{#1}}
\newcommand{\VariableTok}[1]{\textcolor[rgb]{0.07,0.07,0.07}{#1}}
\newcommand{\VerbatimStringTok}[1]{\textcolor[rgb]{0.13,0.47,0.30}{#1}}
\newcommand{\WarningTok}[1]{\textcolor[rgb]{0.37,0.37,0.37}{\textit{#1}}}

\providecommand{\tightlist}{%
  \setlength{\itemsep}{0pt}\setlength{\parskip}{0pt}}\usepackage{longtable,booktabs,array}
\usepackage{calc} % for calculating minipage widths
% Correct order of tables after \paragraph or \subparagraph
\usepackage{etoolbox}
\makeatletter
\patchcmd\longtable{\par}{\if@noskipsec\mbox{}\fi\par}{}{}
\makeatother
% Allow footnotes in longtable head/foot
\IfFileExists{footnotehyper.sty}{\usepackage{footnotehyper}}{\usepackage{footnote}}
\makesavenoteenv{longtable}
\usepackage{graphicx}
\makeatletter
\def\maxwidth{\ifdim\Gin@nat@width>\linewidth\linewidth\else\Gin@nat@width\fi}
\def\maxheight{\ifdim\Gin@nat@height>\textheight\textheight\else\Gin@nat@height\fi}
\makeatother
% Scale images if necessary, so that they will not overflow the page
% margins by default, and it is still possible to overwrite the defaults
% using explicit options in \includegraphics[width, height, ...]{}
\setkeys{Gin}{width=\maxwidth,height=\maxheight,keepaspectratio}
% Set default figure placement to htbp
\makeatletter
\def\fps@figure{htbp}
\makeatother

\KOMAoption{captions}{tableheading}
\makeatletter
\@ifpackageloaded{bookmark}{}{\usepackage{bookmark}}
\makeatother
\makeatletter
\@ifpackageloaded{caption}{}{\usepackage{caption}}
\AtBeginDocument{%
\ifdefined\contentsname
  \renewcommand*\contentsname{Table of contents}
\else
  \newcommand\contentsname{Table of contents}
\fi
\ifdefined\listfigurename
  \renewcommand*\listfigurename{List of Figures}
\else
  \newcommand\listfigurename{List of Figures}
\fi
\ifdefined\listtablename
  \renewcommand*\listtablename{List of Tables}
\else
  \newcommand\listtablename{List of Tables}
\fi
\ifdefined\figurename
  \renewcommand*\figurename{Figure}
\else
  \newcommand\figurename{Figure}
\fi
\ifdefined\tablename
  \renewcommand*\tablename{Table}
\else
  \newcommand\tablename{Table}
\fi
}
\@ifpackageloaded{float}{}{\usepackage{float}}
\floatstyle{ruled}
\@ifundefined{c@chapter}{\newfloat{codelisting}{h}{lop}}{\newfloat{codelisting}{h}{lop}[chapter]}
\floatname{codelisting}{Listing}
\newcommand*\listoflistings{\listof{codelisting}{List of Listings}}
\makeatother
\makeatletter
\makeatother
\makeatletter
\@ifpackageloaded{caption}{}{\usepackage{caption}}
\@ifpackageloaded{subcaption}{}{\usepackage{subcaption}}
\makeatother

\ifLuaTeX
  \usepackage{selnolig}  % disable illegal ligatures
\fi
\usepackage{bookmark}

\IfFileExists{xurl.sty}{\usepackage{xurl}}{} % add URL line breaks if available
\urlstyle{same} % disable monospaced font for URLs
\hypersetup{
  pdftitle={TermiPy},
  pdfauthor={Pratik Kumar},
  colorlinks=true,
  linkcolor={blue},
  filecolor={Maroon},
  citecolor={Blue},
  urlcolor={Blue},
  pdfcreator={LaTeX via pandoc}}


\title{TermiPy}
\author{Pratik Kumar}
\date{2024-10-22}

\begin{document}
\maketitle

\renewcommand*\contentsname{Table of contents}
{
\hypersetup{linkcolor=}
\setcounter{tocdepth}{2}
\tableofcontents
}

\bookmarksetup{startatroot}

\chapter*{Welcome to TermiPy}\label{welcome-to-termipy}
\addcontentsline{toc}{chapter}{Welcome to TermiPy}

\markboth{Welcome to TermiPy}{Welcome to TermiPy}

TermiPy is a lightweight and extensible command-line shell designed for
users who want a simplified, custom terminal interface. It provides core
terminal features like directory navigation, file management, and
command execution, making it easy to interact with your file system.
Built to be minimal and efficient, TermiPy is cross-platform, running on
Linux, macOS, and Windows.

\href{https://badge.fury.io/py/termipy}{\includesvg{index_files/mediabag/termipy.svg}}
\href{https://opensource.org/licenses/MIT}{\includesvg{index_files/mediabag/License-MIT-yellow.svg}}
\href{https://pypi.org/project/termipy/}{\includesvg{index_files/mediabag/termipy1.svg}}

\begin{verbatim}
████████╗███████╗██████╗ ███╗   ███╗██╗██████╗ ██╗   ██╗
╚══██╔══╝██╔════╝██╔══██╗████╗ ████║██║██╔══██╗╚██╗ ██╔╝
   ██║   █████╗  ██████╔╝██╔████╔██║██║██████╔╝ ╚████╔╝ 
   ██║   ██╔══╝  ██╔══██╗██║╚██╔╝██║██║██╔═══╝   ╚██╔╝  
   ██║   ███████╗██║  ██║██║ ╚═╝ ██║██║██║        ██║   
   ╚═╝   ╚══════╝╚═╝  ╚═╝╚═╝     ╚═╝╚═╝╚═╝        ╚═╝   
\end{verbatim}

\section*{Package Goals}\label{package-goals}
\addcontentsline{toc}{section}{Package Goals}

\markright{Package Goals}

\begin{itemize}
\tightlist
\item
  Simplicity: Provide essential shell functionality in a streamlined
  way.
\item
  Extensibility: Offer a flexible base for customization and expansion.
\item
  Cross-Platform Compatibility: Ensure functionality across major
  operating systems.
\end{itemize}

\section*{Key Features}\label{key-features}
\addcontentsline{toc}{section}{Key Features}

\markright{Key Features}

\begin{enumerate}
\def\labelenumi{\arabic{enumi}.}
\tightlist
\item
  File Management: Easily navigate directories, list files, and perform
  file operations.
\item
  Command Execution: Run shell commands directly within TermiPy.
\item
  Resource Monitoring: Check system stats like CPU, memory, and disk
  usage.
\item
  Environment Setup: Quickly configure Python and R environments.
\end{enumerate}

\section*{Installation \& Usage}\label{installation-usage}
\addcontentsline{toc}{section}{Installation \& Usage}

\markright{Installation \& Usage}

Install via pip and run using Python 3 to get started quickly. Visit
Getting Started for more details.

\section*{Contributions}\label{contributions}
\addcontentsline{toc}{section}{Contributions}

\markright{Contributions}

We welcome contributions! Check out our issues page to get involved.

\section*{License}\label{license}
\addcontentsline{toc}{section}{License}

\markright{License}

TermiPy is licensed under the MIT License.

\bookmarksetup{startatroot}

\chapter*{Getting Started}\label{getting-started}
\addcontentsline{toc}{chapter}{Getting Started}

\markboth{Getting Started}{Getting Started}

This guide will help you install TermiPy and get familiar with its basic
usage.

\section*{Installation}\label{installation}
\addcontentsline{toc}{section}{Installation}

\markright{Installation}

You can install TermiPy using pip, the Python package installer:

\begin{Shaded}
\begin{Highlighting}[]
\ExtensionTok{pip}\NormalTok{ install termipy}
\end{Highlighting}
\end{Shaded}

Visit PyPi for more releases - https://pypi.org/project/termipy/

\section*{Running TermiPy}\label{running-termipy}
\addcontentsline{toc}{section}{Running TermiPy}

\markright{Running TermiPy}

After installation, you can start TermiPy by running:

\begin{Shaded}
\begin{Highlighting}[]
\ExtensionTok{termipy}
\end{Highlighting}
\end{Shaded}

If you encounter any PATH issues, you can use:

\begin{Shaded}
\begin{Highlighting}[]
\VariableTok{PATH}\OperatorTok{=}\StringTok{"/usr/bin:/usr/local/bin"} \ExtensionTok{termipy}
\end{Highlighting}
\end{Shaded}

\section*{Basic Usage}\label{basic-usage}
\addcontentsline{toc}{section}{Basic Usage}

\markright{Basic Usage}

Once TermiPy is running, you'll see the TermiPy prompt:

\begin{Shaded}
\begin{Highlighting}[]
\ExtensionTok{@termipy} \OperatorTok{\textgreater{}\textgreater{}}
\end{Highlighting}
\end{Shaded}

You can now start entering commands. Here are some basic commands to get
you started:

\begin{enumerate}
\def\labelenumi{\arabic{enumi}.}
\item
  \textbf{echo \texttt{\textless{}message\textgreater{}}}: Print a
  message to the terminal

\begin{Shaded}
\begin{Highlighting}[]
\ExtensionTok{@termipy} \OperatorTok{\textgreater{}\textgreater{}}\NormalTok{ echo Hello, TermiPy!}
\end{Highlighting}
\end{Shaded}
\item
  \textbf{getwd} or \textbf{ls}: Get current working directory

\begin{Shaded}
\begin{Highlighting}[]
\ExtensionTok{@termipy} \OperatorTok{\textgreater{}\textgreater{}}\NormalTok{ getwd}
\end{Highlighting}
\end{Shaded}
\item
  \textbf{setwd \texttt{\textless{}directory\textgreater{}}}: Change
  directory

\begin{Shaded}
\begin{Highlighting}[]
\ExtensionTok{@termipy} \OperatorTok{\textgreater{}\textgreater{}}\NormalTok{ setwd /path/to/directory}
\end{Highlighting}
\end{Shaded}
\item
  \textbf{typeof \texttt{\textless{}command\textgreater{}}}: Show
  command type

\begin{Shaded}
\begin{Highlighting}[]
\ExtensionTok{@termipy} \OperatorTok{\textgreater{}\textgreater{}}\NormalTok{ typeof echo}
\end{Highlighting}
\end{Shaded}
\item
  \textbf{clear} (aliases: \textbf{cls}, \textbf{clr}): Clear the screen

\begin{Shaded}
\begin{Highlighting}[]
\ExtensionTok{@termipy} \OperatorTok{\textgreater{}\textgreater{}}\NormalTok{ clear}
\end{Highlighting}
\end{Shaded}
\item
  \textbf{tree {[}directory{]}}: Show directory structure

\begin{Shaded}
\begin{Highlighting}[]
\ExtensionTok{@termipy} \OperatorTok{\textgreater{}\textgreater{}}\NormalTok{ tree}
\end{Highlighting}
\end{Shaded}
\item
  \textbf{help}: Display help information

\begin{Shaded}
\begin{Highlighting}[]
\ExtensionTok{@termipy} \OperatorTok{\textgreater{}\textgreater{}}\NormalTok{ help}
\end{Highlighting}
\end{Shaded}
\item
  \textbf{exit}: Exit TermiPy

\begin{Shaded}
\begin{Highlighting}[]
\ExtensionTok{@termipy} \OperatorTok{\textgreater{}\textgreater{}}\NormalTok{ exit}
\end{Highlighting}
\end{Shaded}
\end{enumerate}

For a full list of available commands, use the \texttt{commands}
command:

\begin{Shaded}
\begin{Highlighting}[]
\ExtensionTok{@termipy} \OperatorTok{\textgreater{}\textgreater{}}\NormalTok{ commands}
\end{Highlighting}
\end{Shaded}

\section*{Next Steps}\label{next-steps}
\addcontentsline{toc}{section}{Next Steps}

\markright{Next Steps}

\begin{itemize}
\tightlist
\item
  Explore \href{file-handling.qmd}{File Handling} operations.
\item
  Learn about \href{setting-environment.qmd}{Setting Up Environments}.
\item
  Explore \href{resource-stats.qmd}{Resource Monitoring}.
\item
  Check out the \href{about.qmd}{About} for more documentation and
  support.
\end{itemize}

\bookmarksetup{startatroot}

\chapter*{File Handling}\label{file-handling}
\addcontentsline{toc}{chapter}{File Handling}

\markboth{File Handling}{File Handling}

TermiPy provides a range of commands for file and directory operations.
This guide covers the most commonly used file handling commands.

\section*{Listing Files and
Directories}\label{listing-files-and-directories}
\addcontentsline{toc}{section}{Listing Files and Directories}

\markright{Listing Files and Directories}

\subsection*{getwd or ls}\label{getwd-or-ls}
\addcontentsline{toc}{subsection}{getwd or ls}

List contents of the current directory

\begin{Shaded}
\begin{Highlighting}[]
\ExtensionTok{@termipy} \OperatorTok{\textgreater{}\textgreater{}}\NormalTok{ getwd}
\ExtensionTok{@termipy} \OperatorTok{\textgreater{}\textgreater{}}\NormalTok{ ls}
\end{Highlighting}
\end{Shaded}

\subsection*{tree {[}directory{]}}\label{tree-directory}
\addcontentsline{toc}{subsection}{tree {[}directory{]}}

Show directory structure

\begin{Shaded}
\begin{Highlighting}[]
\ExtensionTok{@termipy} \OperatorTok{\textgreater{}\textgreater{}}\NormalTok{ tree}
\ExtensionTok{@termipy} \OperatorTok{\textgreater{}\textgreater{}}\NormalTok{ tree /path/to/directory}
\end{Highlighting}
\end{Shaded}

\section*{Navigating Directories}\label{navigating-directories}
\addcontentsline{toc}{section}{Navigating Directories}

\markright{Navigating Directories}

\subsection*{\texorpdfstring{setwd
\texttt{\textless{}directory\textgreater{}}}{setwd \textless directory\textgreater{}}}\label{setwd-directory}
\addcontentsline{toc}{subsection}{setwd
\texttt{\textless{}directory\textgreater{}}}

Change the current working directory

\begin{Shaded}
\begin{Highlighting}[]
\ExtensionTok{@termipy} \OperatorTok{\textgreater{}\textgreater{}}\NormalTok{ setwd /path/to/directory}
\end{Highlighting}
\end{Shaded}

\section*{Creating Files and
Directories}\label{creating-files-and-directories}
\addcontentsline{toc}{section}{Creating Files and Directories}

\markright{Creating Files and Directories}

\subsection*{\texorpdfstring{create
\texttt{\textless{}path\textgreater{}}}{create \textless path\textgreater{}}}\label{create-path}
\addcontentsline{toc}{subsection}{create
\texttt{\textless{}path\textgreater{}}}

Create a new file or directory

\begin{Shaded}
\begin{Highlighting}[]
\ExtensionTok{@termipy} \OperatorTok{\textgreater{}\textgreater{}}\NormalTok{ create new\_file.txt}
\ExtensionTok{@termipy} \OperatorTok{\textgreater{}\textgreater{}}\NormalTok{ create new\_directory/}
\end{Highlighting}
\end{Shaded}

\section*{Deleting Files and
Directories}\label{deleting-files-and-directories}
\addcontentsline{toc}{section}{Deleting Files and Directories}

\markright{Deleting Files and Directories}

\subsection*{\texorpdfstring{delete
\texttt{\textless{}path\textgreater{}}}{delete \textless path\textgreater{}}}\label{delete-path}
\addcontentsline{toc}{subsection}{delete
\texttt{\textless{}path\textgreater{}}}

Delete a file or directory

\begin{Shaded}
\begin{Highlighting}[]
\ExtensionTok{@termipy} \OperatorTok{\textgreater{}\textgreater{}}\NormalTok{ delete old\_file.txt}
\ExtensionTok{@termipy} \OperatorTok{\textgreater{}\textgreater{}}\NormalTok{ delete old\_directory/}
\end{Highlighting}
\end{Shaded}

\section*{Renaming Files and
Directories}\label{renaming-files-and-directories}
\addcontentsline{toc}{section}{Renaming Files and Directories}

\markright{Renaming Files and Directories}

\subsection*{\texorpdfstring{rename
\texttt{\textless{}old\textgreater{}}
\texttt{\textless{}new\textgreater{}}}{rename \textless old\textgreater{} \textless new\textgreater{}}}\label{rename-old-new}
\addcontentsline{toc}{subsection}{rename
\texttt{\textless{}old\textgreater{}}
\texttt{\textless{}new\textgreater{}}}

Rename a file or directory

\begin{Shaded}
\begin{Highlighting}[]
\ExtensionTok{@termipy} \OperatorTok{\textgreater{}\textgreater{}}\NormalTok{ rename old\_name.txt new\_name.txt}
\end{Highlighting}
\end{Shaded}

\section*{Searching for Files}\label{searching-for-files}
\addcontentsline{toc}{section}{Searching for Files}

\markright{Searching for Files}

\subsection*{\texorpdfstring{search
\texttt{\textless{}filename\textgreater{}}}{search \textless filename\textgreater{}}}\label{search-filename}
\addcontentsline{toc}{subsection}{search
\texttt{\textless{}filename\textgreater{}}}

Search for a file in the current directory and subdirectories

\begin{Shaded}
\begin{Highlighting}[]
\ExtensionTok{@termipy} \OperatorTok{\textgreater{}\textgreater{}}\NormalTok{ search important\_doc.pdf}
\end{Highlighting}
\end{Shaded}

\section*{File Permissions}\label{file-permissions}
\addcontentsline{toc}{section}{File Permissions}

\markright{File Permissions}

\subsection*{\texorpdfstring{permissions
\texttt{\textless{}file\textgreater{}}}{permissions \textless file\textgreater{}}}\label{permissions-file}
\addcontentsline{toc}{subsection}{permissions
\texttt{\textless{}file\textgreater{}}}

Show file permissions

\begin{Shaded}
\begin{Highlighting}[]
\ExtensionTok{@termipy} \OperatorTok{\textgreater{}\textgreater{}}\NormalTok{ permissions myfile.txt}
\end{Highlighting}
\end{Shaded}

\section*{Disk Usage}\label{disk-usage}
\addcontentsline{toc}{section}{Disk Usage}

\markright{Disk Usage}

\subsection*{diskusage {[}path{]}}\label{diskusage-path}
\addcontentsline{toc}{subsection}{diskusage {[}path{]}}

Show disk usage for a specific path or the current directory

\begin{Shaded}
\begin{Highlighting}[]
\ExtensionTok{@termipy} \OperatorTok{\textgreater{}\textgreater{}}\NormalTok{ diskusage}
\ExtensionTok{@termipy} \OperatorTok{\textgreater{}\textgreater{}}\NormalTok{ diskusage /home/user}
\end{Highlighting}
\end{Shaded}

\section*{File Details}\label{file-details}
\addcontentsline{toc}{section}{File Details}

\markright{File Details}

\subsection*{\texorpdfstring{about
\texttt{\textless{}file\textgreater{}}}{about \textless file\textgreater{}}}\label{about-file}
\addcontentsline{toc}{subsection}{about
\texttt{\textless{}file\textgreater{}}}

Show file details

\begin{Shaded}
\begin{Highlighting}[]
\ExtensionTok{@termipy} \OperatorTok{\textgreater{}\textgreater{}}\NormalTok{ about myfile.txt}
\end{Highlighting}
\end{Shaded}

\section*{Tips for File Handling}\label{tips-for-file-handling}
\addcontentsline{toc}{section}{Tips for File Handling}

\markright{Tips for File Handling}

\begin{enumerate}
\def\labelenumi{\arabic{enumi}.}
\tightlist
\item
  Use tab completion to quickly navigate directories and input file
  names.
\item
  When dealing with files or directories with spaces in their names, use
  quotes:
\end{enumerate}

\begin{Shaded}
\begin{Highlighting}[]
\ExtensionTok{@termipy} \OperatorTok{\textgreater{}\textgreater{}}\NormalTok{ setwd }\StringTok{"My Documents"}
\end{Highlighting}
\end{Shaded}

\begin{enumerate}
\def\labelenumi{\arabic{enumi}.}
\setcounter{enumi}{2}
\tightlist
\item
  Be cautious when using the \texttt{delete} command, as it permanently
  removes files and directories.
\item
  Use the \texttt{tree} command with a depth parameter to limit the
  levels shown:
\end{enumerate}

\begin{Shaded}
\begin{Highlighting}[]
\ExtensionTok{@termipy} \OperatorTok{\textgreater{}\textgreater{}}\NormalTok{ tree }\AttributeTok{{-}L}\NormalTok{ 2}
\end{Highlighting}
\end{Shaded}

\bookmarksetup{startatroot}

\chapter*{Setting Up Environments}\label{setting-up-environments}
\addcontentsline{toc}{chapter}{Setting Up Environments}

\markboth{Setting Up Environments}{Setting Up Environments}

TermiPy provides commands to easily set up Python and R environments.
This guide will walk you through the process of creating and managing
these environments.

\section*{Setting Up a Python
Environment}\label{setting-up-a-python-environment}
\addcontentsline{toc}{section}{Setting Up a Python Environment}

\markright{Setting Up a Python Environment}

TermiPy allows you to create Python virtual environments using the
\texttt{setpyenv} command.

\subsection*{Syntax}\label{syntax}
\addcontentsline{toc}{subsection}{Syntax}

\begin{Shaded}
\begin{Highlighting}[]
\ExtensionTok{setpyenv} \PreprocessorTok{[}\SpecialStringTok{name}\PreprocessorTok{]} \PreprocessorTok{[}\SpecialStringTok{version}\PreprocessorTok{]}
\end{Highlighting}
\end{Shaded}

\begin{itemize}
\tightlist
\item
  \texttt{name}: The name of the virtual environment (optional)
\item
  \texttt{version}: The Python version to use (optional)
\end{itemize}

\subsection*{Examples}\label{examples}
\addcontentsline{toc}{subsection}{Examples}

\begin{enumerate}
\def\labelenumi{\arabic{enumi}.}
\tightlist
\item
  Create a virtual environment with default name and Python version:
\end{enumerate}

\begin{Shaded}
\begin{Highlighting}[]
\ExtensionTok{@termipy} \OperatorTok{\textgreater{}\textgreater{}}\NormalTok{ setpyenv}
\end{Highlighting}
\end{Shaded}

\begin{enumerate}
\def\labelenumi{\arabic{enumi}.}
\setcounter{enumi}{1}
\tightlist
\item
  Create a virtual environment with a specific name:
\end{enumerate}

\begin{Shaded}
\begin{Highlighting}[]
\ExtensionTok{@termipy} \OperatorTok{\textgreater{}\textgreater{}}\NormalTok{ setpyenv myproject}
\end{Highlighting}
\end{Shaded}

\begin{enumerate}
\def\labelenumi{\arabic{enumi}.}
\setcounter{enumi}{2}
\tightlist
\item
  Create a virtual environment with a specific name and Python version:
\end{enumerate}

\begin{Shaded}
\begin{Highlighting}[]
\ExtensionTok{@termipy} \OperatorTok{\textgreater{}\textgreater{}}\NormalTok{ setpyenv myproject 3.9}
\end{Highlighting}
\end{Shaded}

\section*{Setting Up an R
Environment}\label{setting-up-an-r-environment}
\addcontentsline{toc}{section}{Setting Up an R Environment}

\markright{Setting Up an R Environment}

You can initialize an R environment using the \texttt{setrenv} command.

\subsection*{Syntax}\label{syntax-1}
\addcontentsline{toc}{subsection}{Syntax}

\begin{Shaded}
\begin{Highlighting}[]
\ExtensionTok{setrenv} \PreprocessorTok{[}\SpecialStringTok{name}\PreprocessorTok{]}
\end{Highlighting}
\end{Shaded}

\begin{itemize}
\tightlist
\item
  \texttt{name}: The name of the R environment (optional)
\end{itemize}

\subsection*{Example}\label{example}
\addcontentsline{toc}{subsection}{Example}

Create an R environment:

\begin{Shaded}
\begin{Highlighting}[]
\ExtensionTok{@termipy} \OperatorTok{\textgreater{}\textgreater{}}\NormalTok{ setrenv myRproject}
\end{Highlighting}
\end{Shaded}

\section*{Managing Environments}\label{managing-environments}
\addcontentsline{toc}{section}{Managing Environments}

\markright{Managing Environments}

Once you've created an environment, you can use standard Python or R
commands to manage packages and run scripts within that environment.

\subsection*{Activating Environments}\label{activating-environments}
\addcontentsline{toc}{subsection}{Activating Environments}

TermiPy automatically activates the environment upon creation. To switch
between environments, you can use the \texttt{setwd} command to navigate
to the environment's directory.

\subsection*{Installing Packages}\label{installing-packages}
\addcontentsline{toc}{subsection}{Installing Packages}

Use pip for Python environments:

\begin{Shaded}
\begin{Highlighting}[]
\ExtensionTok{@termipy} \OperatorTok{\textgreater{}\textgreater{}}\NormalTok{ pip install package\_name}
\end{Highlighting}
\end{Shaded}

Use install.packages() for R environments:

\begin{Shaded}
\begin{Highlighting}[]
\ExtensionTok{@termipy} \OperatorTok{\textgreater{}\textgreater{}}\NormalTok{ R }\AttributeTok{{-}e} \StringTok{"install.packages(\textquotesingle{}package\_name\textquotesingle{})"}
\end{Highlighting}
\end{Shaded}

\section*{Best Practices}\label{best-practices}
\addcontentsline{toc}{section}{Best Practices}

\markright{Best Practices}

\begin{enumerate}
\def\labelenumi{\arabic{enumi}.}
\tightlist
\item
  Use descriptive names for your environments to easily identify their
  purpose.
\item
  Create separate environments for different projects to avoid package
  conflicts.
\item
  Regularly update your environments to ensure you're using the latest
  package versions.
\item
  Document the packages and versions used in your project for
  reproducibility.
\end{enumerate}

\section*{Troubleshooting}\label{troubleshooting}
\addcontentsline{toc}{section}{Troubleshooting}

\markright{Troubleshooting}

If you encounter any issues while setting up environments, make sure:

\begin{enumerate}
\def\labelenumi{\arabic{enumi}.}
\tightlist
\item
  You have the required Python or R versions installed on your system.
\item
  You have the necessary permissions to create directories and install
  packages.
\item
  Your PATH environment variable is correctly set.
\item
  For R environments, ensure that R is properly installed and accessible
  from the command line.
\end{enumerate}

\bookmarksetup{startatroot}

\chapter*{Resource Monitoring}\label{resource-monitoring}
\addcontentsline{toc}{chapter}{Resource Monitoring}

\markboth{Resource Monitoring}{Resource Monitoring}

TermiPy provides a powerful resource monitoring feature that allows you
to track system resource usage in real-time. This guide will explain how
to use the resource monitoring commands and interpret the output.

\section*{Using the Resource Monitoring
Command}\label{using-the-resource-monitoring-command}
\addcontentsline{toc}{section}{Using the Resource Monitoring Command}

\markright{Using the Resource Monitoring Command}

To start monitoring system resources, use one of the following commands:

\begin{Shaded}
\begin{Highlighting}[]
\ExtensionTok{@termipy} \OperatorTok{\textgreater{}\textgreater{}}\NormalTok{ resource}
\end{Highlighting}
\end{Shaded}

or

\begin{Shaded}
\begin{Highlighting}[]
\ExtensionTok{@termipy} \OperatorTok{\textgreater{}\textgreater{}}\NormalTok{ resources}
\end{Highlighting}
\end{Shaded}

or

\begin{Shaded}
\begin{Highlighting}[]
\ExtensionTok{@termipy} \OperatorTok{\textgreater{}\textgreater{}}\NormalTok{ stats}
\end{Highlighting}
\end{Shaded}

These commands are aliases and will all trigger the same resource
monitoring functionality.

\section*{Understanding the Output}\label{understanding-the-output}
\addcontentsline{toc}{section}{Understanding the Output}

\markright{Understanding the Output}

The resource monitoring feature provides information on five key areas:

\begin{enumerate}
\def\labelenumi{\arabic{enumi}.}
\tightlist
\item
  CPU Usage
\item
  Memory Usage
\item
  Disk Usage
\item
  Network Usage
\item
  Process Usage (Top 5 CPU-consuming processes)
\end{enumerate}

\subsection*{CPU Usage}\label{cpu-usage}
\addcontentsline{toc}{subsection}{CPU Usage}

This section displays the current CPU usage as a percentage. The
percentage is color-coded for quick interpretation:

\begin{itemize}
\tightlist
\item
  Green: 0-69\% (Low to moderate usage)
\item
  Yellow: 70-89\% (High usage)
\item
  Red: 90-100\% (Very high usage)
\end{itemize}

Example output:

\begin{Shaded}
\begin{Highlighting}[]
\ExtensionTok{┌───────────────────────────────────────────┐}
\ExtensionTok{│}\NormalTok{ CPU Usage                                 │}
\ExtensionTok{├───────────────────────────────────────────┤}
\ExtensionTok{│}\NormalTok{ CPU Usage: 3.5\%                           │}
\ExtensionTok{└───────────────────────────────────────────┘}
\end{Highlighting}
\end{Shaded}

\subsection*{Memory Usage}\label{memory-usage}
\addcontentsline{toc}{subsection}{Memory Usage}

This section shows the current memory usage, including:

\begin{itemize}
\tightlist
\item
  Memory Usage Percentage (color-coded like CPU usage)
\item
  Total Memory
\item
  Available Memory
\end{itemize}

Example output:

\begin{Shaded}
\begin{Highlighting}[]
\ExtensionTok{┌───────────────────────────────────────────┐}
\ExtensionTok{│}\NormalTok{ Memory Usage                              │}
\ExtensionTok{├───────────────────────────────────────────┤}
\ExtensionTok{│}\NormalTok{ Memory Usage: 30.0\%                       │}
\ExtensionTok{│}\NormalTok{ Total Memory: 7.74 GB                     │}
\ExtensionTok{│}\NormalTok{ Available Memory: 5.42 GB                 │}
\ExtensionTok{└───────────────────────────────────────────┘}
\end{Highlighting}
\end{Shaded}

\subsection*{Disk Usage}\label{disk-usage-1}
\addcontentsline{toc}{subsection}{Disk Usage}

This section displays information about disk usage:

\begin{itemize}
\tightlist
\item
  Disk Usage Percentage (color-coded)
\item
  Total Disk Space
\item
  Free Disk Space
\end{itemize}

Example output:

\begin{Shaded}
\begin{Highlighting}[]
\ExtensionTok{┌───────────────────────────────────────────┐}
\ExtensionTok{│}\NormalTok{ Disk Usage                                │}
\ExtensionTok{├───────────────────────────────────────────┤}
\ExtensionTok{│}\NormalTok{ Disk Usage: 38.0\%                         │}
\ExtensionTok{│}\NormalTok{ Total Disk Space: 31.33 GB                │}
\ExtensionTok{│}\NormalTok{ Free Disk Space: 18.43 GB                 │}
\ExtensionTok{└───────────────────────────────────────────┘}
\end{Highlighting}
\end{Shaded}

\subsection*{Network Usage}\label{network-usage}
\addcontentsline{toc}{subsection}{Network Usage}

This section shows the amount of data sent and received over the
network:

\begin{itemize}
\tightlist
\item
  Network Bytes Sent
\item
  Network Bytes Received
\end{itemize}

Example output:

\begin{Shaded}
\begin{Highlighting}[]
\ExtensionTok{┌───────────────────────────────────────────┐}
\ExtensionTok{│}\NormalTok{ Network Usage                             │}
\ExtensionTok{├───────────────────────────────────────────┤}
\ExtensionTok{│}\NormalTok{ Network Bytes Sent: 346.95 MB             │}
\ExtensionTok{│}\NormalTok{ Network Bytes Received: 22058.71 MB       │}
\ExtensionTok{└───────────────────────────────────────────┘}
\end{Highlighting}
\end{Shaded}

\subsection*{Process Usage}\label{process-usage}
\addcontentsline{toc}{subsection}{Process Usage}

This section lists the top 5 CPU-consuming processes, including:

\begin{itemize}
\tightlist
\item
  Process ID (PID)
\item
  Process Name
\item
  CPU Usage Percentage (color-coded)
\end{itemize}

Example output:

\begin{Shaded}
\begin{Highlighting}[]
\ExtensionTok{┌───────────────────────────────────────────┐}
\ExtensionTok{│}\NormalTok{ Process Usage }\ErrorTok{(}\ExtensionTok{Top{-}5}\KeywordTok{)}                     \ExtensionTok{│}
\ExtensionTok{├───────────────────────────────────────────┤}
\ExtensionTok{│}\NormalTok{ PID: 28520, Name: node, CPU: 1.0\%         │}
\ExtensionTok{│}\NormalTok{ PID: 1, Name: docker{-}init, CPU: 0.0\%      │}
\ExtensionTok{│}\NormalTok{ PID: 7, Name: sleep, CPU: 0.0\%            │}
\ExtensionTok{│}\NormalTok{ PID: 35, Name: sshd, CPU: 0.0\%            │}
\ExtensionTok{│}\NormalTok{ PID: 412, Name: sh, CPU: 0.0\%             │}
\ExtensionTok{└───────────────────────────────────────────┘}
\end{Highlighting}
\end{Shaded}

\section*{Continuous Monitoring}\label{continuous-monitoring}
\addcontentsline{toc}{section}{Continuous Monitoring}

\markright{Continuous Monitoring}

The resource monitoring feature updates every second, providing a
real-time view of your system's resource usage. The display is refreshed
automatically, clearing the screen and showing the latest information.

To stop the monitoring, press \texttt{Ctrl+C}. You'll see a message
confirming that the monitoring has stopped:

\begin{Shaded}
\begin{Highlighting}[]
\ExtensionTok{Stopped}\NormalTok{ monitoring resource usage.}
\end{Highlighting}
\end{Shaded}

\section*{Use Cases}\label{use-cases}
\addcontentsline{toc}{section}{Use Cases}

\markright{Use Cases}

\begin{enumerate}
\def\labelenumi{\arabic{enumi}.}
\tightlist
\item
  \textbf{Performance Troubleshooting}: Identify resource-intensive
  processes that might be causing system slowdowns.
\item
  \textbf{Capacity Planning}: Monitor resource usage over time to
  determine if you need to upgrade your hardware.
\item
  \textbf{Application Testing}: Monitor resource usage while testing
  applications to ensure they're not consuming excessive resources.
\item
  \textbf{System Health Checks}: Regularly check your system's resource
  usage to ensure it's operating within expected parameters.
\end{enumerate}

\section*{Tips for Using Resource
Monitoring}\label{tips-for-using-resource-monitoring}
\addcontentsline{toc}{section}{Tips for Using Resource Monitoring}

\markright{Tips for Using Resource Monitoring}

\begin{enumerate}
\def\labelenumi{\arabic{enumi}.}
\tightlist
\item
  Run the monitoring command before and after starting a
  resource-intensive task to see its impact on your system.
\item
  Use the \textbf{color-coding} to quickly identify areas of concern
  (yellow and red indicators).
\item
  Pay attention to the top CPU-consuming processes to identify any
  unexpected or misbehaving applications.
\item
  Monitor network usage to detect unusual spikes in data transfer, which
  could indicate network issues or unauthorized data transfers.
\end{enumerate}

\bookmarksetup{startatroot}

\chapter*{About}\label{about}
\addcontentsline{toc}{chapter}{About}

\markboth{About}{About}

TermiPy is a simple command-line shell providing essential shell
functionalities such as directory navigation, file listing, and command
execution. It is designed to be minimal, lightweight, and highly
extensible for users looking to interact with their file systems through
a custom terminal interface.

\section*{Why TermiPy?}\label{why-termipy}
\addcontentsline{toc}{section}{Why TermiPy?}

\markright{Why TermiPy?}

\begin{itemize}
\tightlist
\item
  \textbf{Simplicity}: TermiPy focuses on core functionalities without
  unnecessary complexity.
\item
  \textbf{Extensibility}: Easily add new commands and features to suit
  your needs.
\item
  \textbf{Cross-Platform}: Works seamlessly on Linux, macOS, and
  Windows.
\item
  \textbf{Python-Based}: Leverages the power and flexibility of Python.
\end{itemize}

\section*{Key Features}\label{key-features-1}
\addcontentsline{toc}{section}{Key Features}

\markright{Key Features}

\begin{enumerate}
\def\labelenumi{\arabic{enumi}.}
\tightlist
\item
  \textbf{File and Directory Operations}:

  \begin{itemize}
  \tightlist
  \item
    Navigate directories
  \item
    List files
  \item
    Create, delete, and rename files/directories
  \end{itemize}
\item
  \textbf{Command Execution}:

  \begin{itemize}
  \tightlist
  \item
    Execute shell commands directly through TermiPy
  \end{itemize}
\item
  \textbf{System Resource Monitoring}:

  \begin{itemize}
  \tightlist
  \item
    View real-time CPU, memory, disk, and network usage statistics
  \end{itemize}
\item
  \textbf{Environment Setup}:

  \begin{itemize}
  \tightlist
  \item
    Set up Python and R environments with ease
  \end{itemize}
\item
  \textbf{Cross-Platform Compatibility}:

  \begin{itemize}
  \tightlist
  \item
    Works on Linux, macOS, and Windows
  \end{itemize}
\end{enumerate}

\section*{Author}\label{author}
\addcontentsline{toc}{section}{Author}

\markright{Author}

TermiPy is created and maintained by Pratik Kumar.

\begin{itemize}
\tightlist
\item
  Website: \href{https://pr2tik1.github.io/}{pr2tik1}
\item
  Twitter: \href{https://twitter.com/pr2tik1}{@pr2tik1}
\item
  GitHub: \href{https://github.com/pr2tik1}{@pr2tik1}
\item
  LinkedIn:
  \href{https://www.linkedin.com/in/pratik-kumar/}{@pratik-kumar}
\end{itemize}

\section*{License}\label{license-1}
\addcontentsline{toc}{section}{License}

\markright{License}

TermiPy is open-source software licensed under the MIT License. For more
details, see the
\href{https://github.com/pr2tik1/termipy/blob/main/LICENSE}{LICENSE}
file in the project repository.

\section*{Contributing}\label{contributing}
\addcontentsline{toc}{section}{Contributing}

\markright{Contributing}

Contributions, issues, and feature requests are welcome! Feel free to
check the \href{https://github.com/pr2tik1/termipy/issues}{issues page}
on GitHub.

\section*{Show Your Support}\label{show-your-support}
\addcontentsline{toc}{section}{Show Your Support}

\markright{Show Your Support}

If you find TermiPy helpful, please give it a star on GitHub!




\end{document}
